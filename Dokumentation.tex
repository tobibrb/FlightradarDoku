\documentclass[ngerman, oneside]{article}

\usepackage[latin1]{inputenc} % Zeichensatz, erm�glicht die direkte Eingabe von Umlauten im Editor
\usepackage[ngerman]{babel}   % Silbentrennung nach der neuen deutschen Rechtschreibung, z.B.: Sys-tem
\usepackage{setspace}
\usepackage[footnote]{acronym}%f�r Abk�rzungsverzeichnis
\usepackage[pdftex]{graphicx} % Einbindung von Grafiken (pdf, png, jpg)
\usepackage{pdfpages}         % f�r die Einbindung kompletter pdf-*Seiten*
\usepackage{float}						%um Bilder genau zu positionieren: ... \begin{figure}[H] ...
\usepackage{xcolor}						%f�r Bunt!
\usepackage{varioref}%f�r interne verweise
\usepackage[
colorlinks, %farbige links
linkcolor={blue!50!black},
citecolor={blue!50!black},
filecolor={blue!50!black},
pagecolor={blue!50!black},
urlcolor={blue!50!black},
hyperfootnodes
]{hyperref}%f�r interne verweise
\usepackage{cleveref}%f�r interne verweise
\usepackage[babel,german=quotes]{csquotes} %deutsche anf�hrungszeichen
%\usepackage[showframe]{geometry} %debug
\usepackage[numbers]{natbib}%f�r Literaturverzeichnis
%%	WICHTIG:
%%-Bei Labels und Dateinamen Umlaute vermeiden
%%-Ausgelagerte Texte mit ANSI kodiert abspeichern
\usepackage[hyphens]{url}
\usepackage{hyperref}
\usepackage{geometry}
\geometry{a4paper, top=25mm, left=40mm, right=25mm, bottom=30mm, headsep=10mm, footskip=12mm}

\begin{document}

\begin{titlepage}
\begin{center}
\textbf{\Huge Kurzdokumentation\vspace{2,5mm}\\\large{f�r}\\\vspace{2,5mm}\Huge A.B.C. Alert}\\
\vspace{5mm}
Version 1.0

		\vspace{\fill}
		

		\begin{table}[hp]
			\centering
			\renewcommand{\arraystretch}{1,5}%gr��erer Zeilenabstand
				\begin{tabular}{ l c r }%linke spalte linksb�ndig | mittlere Spalte zentriert | rechte Spalte rechtsb�ndig
					Anders,Toni & Student & anderst@fh-brandenburg.de \\ 
					Bartz,Tobias & Student & bartz@fh-brandenburg.de \\ 
					Christ,Colin & Student & christ@fh-brandenburg.de \\
				\end{tabular}
		\end{table}
		\\
		\setstretch{1.5}
		Dozent: Preu�, Thomas\\
		Lehrveranstaltung: Systemintegration
		
		
		Datum: 15.12.2015

		\end{center}
\end{titlepage} 

\tableofcontents
\clearpage

\section{Einf�hrung}
\label{sec:Einfuehrung}

\subsection{Aufgabenstellung}
\label{sec:Aufgabenstellung}


\clearpage

\section{Architektur�berblick}
\label{Architekturueberblick}
Dieses Kapitel soll einen kurzen �berblick �ber die vorhandene Architektur geben.

\subsection{Technologien}
\label{Technologien}
\subsubsection{Gradle}
\subsubsection{Travis-CI}
\subsubsection{Docker}

\subsection{Flightradar24.com API}
\label{FlightradarAPI}

\subsection{AWS-Dienste}
\label{AWSDienste}

\subsubsection{S3}
\subsubsection{Lambda}
\subsubsection{SES}
\subsubsection{DynamoDB}

\clearpage

\section{Funktionalit�t des Systems}
\label{sec:FunktionalitaetDesSystems}
Da in dieser Anwendung verschiedene AWS-Dienste verwendet werden (siehe Kapitel \ref{AWSDienste}), ist es notwendig die entsprechenden Credentials zu hinterlegen. Dadurch funktioniert das Ausf�hren der Anwendung auf einen Host ohne diese zu einem Fehler. In dem zur Verf�gung gestellten Docker Image sind diese Credentials bereits hinterlegt. Daher ist ein Ausf�hren der Anwendung nur mithilfe des Docker Images m�glich (siehe Kapitel \ref{Docker}).

Die Anwendung wird �ber eine \verb+REST+-Schnittstelle verwendet. Im folgenden ist aufgef�hrt wie die Anwendung zu bedienen ist. Die Anwendung wird, wie im Kapitel \ref{Docker} beschrieben, gestartet. Nach dem Start kann �ber folgende \verb+URL+ eine E-Mail Adresse bei dem System angemeldet werden.
\begin{center}
	\verb+http://<host>:8080/flight-radar/sub-email?email+
	\verb+=max.mustermann@muster-mail.de[&airport=TXL]+
\end{center}}
Der Queryparameter \verb+airport+ ist optional, d.h. wird er nicht angegeben bekommt der Nutzer nach seiner Validierung alle Fl�ge, welche sich zurzeit �ber Berlin und Brandenburg befinden. Wird ein \verb+airport+ angegeben so bekommt der Nutzer sp�ter nur jene Fl�ge, welche den entsprechenden Flughafen als Start oder Ziel haben. Als Wert f�r diesen Parameter ist die internationale Kennung der Flugh�fen zu benutzen z.B. TXL f�r Berlin Tegel oder SXF f�r Berlin Sch�nefeld.
Nach dem der Nutzer seine Email Adresse registriert hat, bekommt er eine E-Mail an jene Adresse. In dieser befindet sich dann eine Hyperlink f�r die Validierung der Adresse. Der Linkt hat folgendes Format:
\begin{center}
	\verb+http://<host>:8080/flight-radar/sub-email/+
	\verb+validate?uuid=02aff3e4-4223-434f-bc7c-8ff7bb57d90b+
\end{center} 
wobei der Parameter \verb+uuid+ vom System erstellt wird und keine weitere Bedeutung f�r den Nutzer hat. Nach der erfolgreichen Registrierung erh�lt der Nutzer ca. alle 30 Minuten eine Email, wenn f�r ihn relevante Fl�ge in der Datenbank enthalten sind.
Dem Nutzer ist es m�glich die Auswahl der Flugh�fen nachtr�glich zu �ndern. Hier f�r ist es notwendig das er die Registrierung nochmals durchl�uft nur dies mal mit anderen Parametern f�r \verb+airport+. Es ist m�glich auch mehrere Flugh�fen anzugeben. Z.B. so:
\begin{center}
	\verb+...&airport=TXL&airport=SXF&airport=AMS+
\end{center}}
Sollte der Nutzer kein interesse mehr an den Emails haben, so kann er �ber den folgenden Link den Dienst abbestellen.
\begin{center}
	\verb+http://<host>:8080/flight-radar/unsub-email?email=max.mustermann@muster-mail.de
\end{center}}


\end{document}