\documentclass[ngerman, oneside]{article}

\usepackage[latin1]{inputenc} % Zeichensatz, erm�glicht die direkte Eingabe von Umlauten im Editor
\usepackage[ngerman]{babel}   % Silbentrennung nach der neuen deutschen Rechtschreibung, z.B.: Sys-tem
\usepackage{setspace}
\usepackage[footnote]{acronym}%f�r Abk�rzungsverzeichnis
\usepackage[pdftex]{graphicx} % Einbindung von Grafiken (pdf, png, jpg)
\usepackage{pdfpages}         % f�r die Einbindung kompletter pdf-*Seiten*
\usepackage{float}						%um Bilder genau zu positionieren: ... \begin{figure}[H] ...
\usepackage{xcolor}						%f�r Bunt!
\usepackage{varioref}%f�r interne verweise
\usepackage[
colorlinks, %farbige links
linkcolor={blue!50!black},
citecolor={blue!50!black},
filecolor={blue!50!black},
pagecolor={blue!50!black},
urlcolor={blue!50!black},
hyperfootnodes
]{hyperref}%f�r interne verweise
\usepackage{cleveref}%f�r interne verweise
\usepackage[babel,german=quotes]{csquotes} %deutsche anf�hrungszeichen
%\usepackage[showframe]{geometry} %debug
\usepackage[numbers]{natbib}%f�r Literaturverzeichnis
%%	WICHTIG:
%%-Bei Labels und Dateinamen Umlaute vermeiden
%%-Ausgelagerte Texte mit ANSI kodiert abspeichern
\begin{document}

\begin{titlepage}
\begin{center}
\textbf{\Huge Kurzdokumentation\vspace{2,5mm}\\\large{f�r}\\\vspace{2,5mm}\Huge A.B.C. Alert}\\
\vspace{5mm}
Version 1.0

		\vspace{\fill}
		

		\begin{table}[hp]
			\centering
			\renewcommand{\arraystretch}{1,5}%gr��erer Zeilenabstand
				\begin{tabular}{ l c r }%linke spalte linksb�ndig | mittlere Spalte zentriert | rechte Spalte rechtsb�ndig
					Anders, Toni & Student & anderst@fh-brandenburg.de \\ 
					Bartz, Tobias & Student & bartz@fh-brandenburg.de \\ 
					Christ, Colin & Student & christ@fh-brandenburg.de \\
				\end{tabular}
		\end{table}
		\\
		\setstretch{1.5}
		Dozent: Preu�, Thomas\\
		Dozent: Gentsch, Lars\\
		Lehrveranstaltung: Systemintegration
		
		
		Datum: 15.12.2015

		\end{center}
\end{titlepage} 

\tableofcontents
\clearpage
\addcontentsline{toc}{section}{Historie der Dokumentversionen}
\section*{Historie der Dokumentversionen}
\label{sec:HistorieDerDokumentversionen}


\begin{table}[hp]
	\centering
		\begin{tabular}{l c c p{8cm}}
			Version & Datum & Autor/en & �nderungsgrund/Bemerkungen\\\hline
			0.1			& 12.12.15& Bartz & Anlegen und formatieren des Dokumentes\\
			\end{tabular}
	\label{Historie der Dokumentversionen}
\end{table}

\clearpage

\section{Einf�hrung}
\label{sec:Einfuehrung}
Ziel dieser Dokumentation ist eine kurzer �berblick �ber die entstandene Anwendung. Sie soll dem besserem Verst�ndnis der dienen und ein Testen vereinfachen.

\subsection{Aufgabenstellung}
\label{sec:Aufgabenstellung}
Im Rahmen des Semesterprojekts in der Veranstaltung "`Systemintegration"' soll eine Cloud-basierte Anwendung unter Verwendung von CI/CD erstellen. Hierbei 


\clearpage

\section{Architektur�berblick}
\label{Architekturueberblick}
Dieses Kapitel soll einen kurzen �berblick �ber die vorhandene Architektur geben.

\subsection{GitHub}
\label{GitHub}
Bei GitHub handelt es sich um ein auf Git basiertes Versions-Kontroll-System. Es wurde im Rahmen des Semsterprojekts verwendet um das Entwickeln im Team zu vereinfachen und �nderungen besser verfolgen zu k�nnen. Das GitHub-Repository dieses Projektes befindet sich unter:
\begin{center}
	\verb+https://github.com/tobibrb/FlightradarProject+
\end{center}


\subsection{Gradle}
\label{Gradle}
Build-Tools machen das automatische Bauen von Anwendungen einfach. Bei der Auswahl des Build-Tools fiel die Wahl auf Gradle. Dependency-Management und das Erstellen eigener Tasks, die w�hrend des Builds ausgef�hrt werden ist m�glich. So wurde in dieser Anwendung ein Task \verb+integrationTests+ erstellt. Dieser dient zum Ausf�hren der Integration-Tests. Hierbei wird das Schreiben und Lesen auf der Datenbank getestet.

\subsection{Travis CI}
\label{TravisCI}
Als Continuous-Integration-Server wird in dieser Anwendung Travis CI verwendet. Mit Travis CI ist es m�glich GitHub-Repositories automatisch zu bauen und die entsprechenden Tests durchzuf�hren. Die Konfiguration des Builds erfolgt dabei in der \verb+.travis.yml+ Datei. Um ein kontinuierliches Deployen der Anwendung zu erreichen kann Travis CI verwendet werden um automatisch Docker Container zu erstellen. Ist das Kompilieren sowie Testen der Anwendung erfolgreich, erstellt Travis CI den Docker Container. Ein Nachverfolgen der Build-Historie ist unter folgender URL m�glich:
\begin{center}
	\verb+https://travis-ci.org/tobibrb/FlightradarProject+
\end{center}


\subsection{Docker}
\label{Docker}
In dieser Anwendung wird durch Travis CI automatisch ein Docker Container erzeugt. Dieser wird als Image auf der Plattform "`Docker Hub"' zur Verf�gung gestellt. Docker Hub stellt dabei die Registry f�r das Repository bereit. Die aktuellste Version des Docker Images kann mit folgendem Befehl auf einen Host mit Docker heruntergeladen werden:
\begin{center}
	\verb+docker pull cronjoe/flightradar+
\end{center}
Der Docker Container enth�lt eine lauff�hige Version der Anwendung. Der Container kann mit folgendem Befehl gestartet werden:
\begin{center}
	\verb+docker run --rm -i -t -p 8080:8080 cronjoe/flightradar+
\end{center}

\subsection{Flightradar24.com API}
\label{FlightradarAPI}
Bei Flightradar24 handelt es sich um einen Live-Tracking-Service f�r Flugzeuge. Hierzu werden Daten von vielen verschiedenen Diensten zusammen gefasst. Die Daten werden in einer nicht �ffentlichen API zur Verf�gung gestellt. Diese wird von den Produkten und Anwendungen der Flightradar24 AB verwendet um die aktuellen Flugdaten zur Verf�gung zu stellen.

\subsection{AWS-Dienste}
\label{AWSDienste}

\subsubsection{S3}
\label{S3}
\subsubsection{Lambda}
\label{Lambda}
\subsubsection{SES}
\label{SES}
\subsubsection{DynamoDB}
\label{DynamoDB}

\clearpage

\section{Funktionalit�t des Systems}
\label{sec:FunktionalitaetDesSystems}



\end{document}