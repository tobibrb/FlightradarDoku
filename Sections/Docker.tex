In dieser Anwendung wird durch Travis CI automatisch ein Docker Container erzeugt. Dieser wird als Image auf der Plattform "`Docker Hub"' zur Verf�gung gestellt. Docker Hub stellt dabei die Registry f�r das Repository bereit. Die aktuellste Version des Docker Images kann mit folgendem Befehl auf einen Host mit Docker heruntergeladen werden:
\begin{center}
	\verb+docker pull cronjoe/flightradar+
\end{center}
Der Docker Container enth�lt eine lauff�hige Version der Anwendung. Der Container kann mit folgendem Befehl gestartet werden:
\begin{center}
	\verb+docker run --rm -i -t -p 8080:8080 cronjoe/flightradar+
\end{center}
Docker ist eine Linux-basierte Software, die auf dem Hostsystem in einer virtuellen Umgebung Container ausf�hren kann. Die Installation von Docker ist dabei auf vielen Betriebssystemen m�glich. Eine ausf�hrliche Installationsanleitung findet sich hier:
\begin{center}
	\verb+https://docs.docker.com/+
\end{center}
Durch die Verwendung eines Docker Containers ist es m�glich eine komplett lauff�hige Version der Anwendung bereitzustellen. Es w�re z.B. m�glich, den Container beim AWS-Dienst \verb+EC2 Container Service+ zu starten.