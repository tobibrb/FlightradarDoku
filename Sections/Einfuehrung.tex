\section{Einf�hrung}
\label{sec:Einfuehrung}
Ziel dieser Dokumentation ist eine kurzer �berblick �ber die entstandene Anwendung. Sie soll dem besserem Verst�ndnis der dienen und ein Testen vereinfachen.

Im Rahmen des Semesterprojekts in der Veranstaltung "`Systemintegration"' soll eine Cloud-basierte Anwendung unter Verwendung von CI/CD erstellen. Die hierbei gestellten Anforderungen lauten:
\begin{itemize}
	\item Verwendung von mindestens vier AWS-Diensten
	\item Implementation von zwei Testphasen
	\item Pro Testphase mindestens einen Test
	\item Verwendung eines VCS
	\item Verwendung eines CI-Servers
	\item Verpacken und Deployen der Anwendung per Docker
\end{itemize}

Ziel dieses Projektes ist das Erstellen einer Anwendung zum Auswerten der Flugdaten von \verb+http://flightradar24.com/+. Hierzu soll in regelm��igen Abst�nden gepr�ft werden, ob sich aktuell Flugzeuge �ber Berlin/Brandenburg befinden.

Benutzer sollen die M�glichkeit haben sich f�r Aktualisierungen zu registrieren. Diese werden dem Benutzer per E-Mail zur Verf�gung gestellt. Das Aktualisieren der Flugdaten erfolgt mithilfe der internen Flightradar24 API. Diese wird vom Unternehmen normalerweise genutzt um ihren eigenen Anwendung aktuelle Daten zur Verf�gung zu stellen.
