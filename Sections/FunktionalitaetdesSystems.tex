\section{Funktionalit�t des Systems}
\label{sec:FunktionalitaetDesSystems}
Da in dieser Anwendung verschiedene AWS-Dienste verwendet werden (siehe Kapitel \ref{AWSDienste}), ist es notwendig die entsprechenden Credentials zu hinterlegen. Dadurch funktioniert das Ausf�hren der Anwendung auf einen Host ohne diese zu einem Fehler. In dem zur Verf�gung gestellten Docker Image sind diese Credentials bereits hinterlegt. Daher ist ein Ausf�hren der Anwendung nur mithilfe des Docker Images m�glich (siehe Kapitel \ref{Docker}).

Die Anwendung wird �ber eine \verb+REST+-Schnittstelle verwendet. Im folgenden ist aufgef�hrt wie die Anwendung zu bedienen ist. Die Anwendung wird, wie im Kapitel \ref{Docker} beschrieben, gestartet. Nach dem Start kann �ber folgende \verb+URL+ eine E-Mail Adresse bei dem System angemeldet werden.
\begin{center}
	\verb+http://<host>:8080/flight-radar/sub-email?email+
	\verb+=max.mustermann@muster-mail.de[&airport=TXL]+
\end{center}}
Der Queryparameter \verb+airport+ ist optional, d.h. wird er nicht angegeben bekommt der Nutzer nach seiner Validierung alle Fl�ge, welche sich zurzeit �ber Berlin und Brandenburg befinden. Wird ein \verb+airport+ angegeben so bekommt der Nutzer sp�ter nur jene Fl�ge, welche den entsprechenden Flughafen als Start oder Ziel haben. Als Wert f�r diesen Parameter ist die internationale Kennung der Flugh�fen zu benutzen z.B. TXL f�r Berlin Tegel oder SXF f�r Berlin Sch�nefeld.
Nach dem der Nutzer seine Email Adresse registriert hat, bekommt er eine E-Mail an jene Adresse. In dieser befindet sich dann eine Hyperlink f�r die Validierung der Adresse. Der Linkt hat folgendes Format:
\begin{center}
	\verb+http://<host>:8080/flight-radar/sub-email/+
	\verb+validate?uuid=02aff3e4-4223-434f-bc7c-8ff7bb57d90b+
\end{center} 
wobei der Parameter \verb+uuid+ vom System erstellt wird und keine weitere Bedeutung f�r den Nutzer hat. Nach der erfolgreichen Registrierung erh�lt der Nutzer ca. alle 30 Minuten eine Email, wenn f�r ihn relevante Fl�ge in der Datenbank enthalten sind.
Dem Nutzer ist es m�glich die Auswahl der Flugh�fen nachtr�glich zu �ndern. Hier f�r ist es notwendig das er die Registrierung nochmals durchl�uft nur dies mal mit anderen Parametern f�r \verb+airport+. Es ist m�glich auch mehrere Flugh�fen anzugeben. Z.B. so:
\begin{center}
	\verb+...&airport=TXL&airport=SXF&airport=AMS+
\end{center}}
Sollte der Nutzer kein interesse mehr an den Emails haben, so kann er �ber den folgenden Link den Dienst abbestellen.
\begin{center}
	\verb+http://<host>:8080/flight-radar/unsub-email?email=max.mustermann@muster-mail.de
\end{center}}