Als Continuous-Integration-Server wird in dieser Anwendung Travis CI verwendet. Mit Travis CI ist es m�glich GitHub-Repositories automatisch zu bauen und die entsprechenden Tests durchzuf�hren. Die Konfiguration des Builds erfolgt dabei in der \verb+.travis.yml+ Datei. Um ein kontinuierliches Deployen der Anwendung zu erreichen kann Travis CI verwendet werden um automatisch Docker Container zu erstellen. Ist das Kompilieren sowie Testen der Anwendung erfolgreich, erstellt Travis CI den Docker Container. Ein Nachverfolgen der Build-Historie ist unter folgender URL m�glich:
\begin{center}
	\verb+https://travis-ci.org/tobibrb/FlightradarProject+
\end{center}
